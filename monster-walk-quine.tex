% LaTeX Quine: Monster Walk 15 Layers
% Self-constructing document that folds 71→59→47→... symmetries

\documentclass{article}
\usepackage{amsmath}
\usepackage{listings}

\begin{document}

\title{Monster Walk: 15-Layer Quine}
\author{Generated from 196883 Shards}
\maketitle

\section*{Layer 0: Monster (196883 = 71×59×47)}

\begin{equation}
M = \bigotimes_{i=0}^{196883} V_i
\end{equation}

\section*{Layer 1: Split into 71 Shards}

\begin{equation}
\text{Shard}_{71}(i) = i \bmod 71, \quad i \in [0, 196883)
\end{equation}

\section*{Layer 2: Each Shard Splits into 59}

\begin{equation}
\text{Shard}_{59}(i) = \lfloor i / 71 \rfloor \bmod 59
\end{equation}

\section*{Layer 3: Each Splits into 47}

\begin{equation}
\text{Shard}_{47}(i) = \lfloor i / (71 \times 59) \rfloor \bmod 47
\end{equation}

\section*{Layer 4-15: Recursive Folding}

\begin{align}
L_4 &: 47 \to 43 \text{ (next prime)} \\
L_5 &: 43 \to 41 \\
L_6 &: 41 \to 37 \\
L_7 &: 37 \to 31 \\
L_8 &: 31 \to 29 \\
L_9 &: 29 \to 23 \text{ (Paxos nodes)} \\
L_{10} &: 23 \to 19 \\
L_{11} &: 19 \to 17 \\
L_{12} &: 17 \to 13 \\
L_{13} &: 13 \to 11 \\
L_{14} &: 11 \to 7 \\
L_{15} &: 7 \to 5 \to 3 \to 2 \to 1 \to 0
\end{align}

\section*{Quine Property}

This document source code:

\begin{lstlisting}[language=TeX]
% LaTeX Quine: Monster Walk 15 Layers
% Self-constructing document that folds 71→59→47→... symmetries

\documentclass{article}
\usepackage{amsmath}
\usepackage{listings}

\begin{document}

\title{Monster Walk: 15-Layer Quine}
\author{Generated from 196883 Shards}
\maketitle

\section*{Layer 0: Monster (196883 = 71×59×47)}

\begin{equation}
M = \bigotimes_{i=0}^{196883} V_i
\end{equation}

\section*{Layer 1: Split into 71 Shards}

\begin{equation}
\text{Shard}_{71}(i) = i \bmod 71, \quad i \in [0, 196883)
\end{equation}

\section*{Layer 2: Each Shard Splits into 59}

\begin{equation}
\text{Shard}_{59}(i) = \lfloor i / 71 \rfloor \bmod 59
\end{equation}

\section*{Layer 3: Each Splits into 47}

\begin{equation}
\text{Shard}_{47}(i) = \lfloor i / (71 \times 59) \rfloor \bmod 47
\end{equation}

\section*{Layer 4-15: Recursive Folding}

\begin{align}
L_4 &: 47 \to 43 \text{ (next prime)} \\
L_5 &: 43 \to 41 \\
L_6 &: 41 \to 37 \\
L_7 &: 37 \to 31 \\
L_8 &: 31 \to 29 \\
L_9 &: 29 \to 23 \text{ (Paxos nodes)} \\
L_{10} &: 23 \to 19 \\
L_{11} &: 19 \to 17 \\
L_{12} &: 17 \to 13 \\
L_{13} &: 13 \to 11 \\
L_{14} &: 11 \to 7 \\
L_{15} &: 7 \to 5 \to 3 \to 2 \to 1 \to 0
\end{align}

\section*{Quine Property}

This document source code:

\begin{lstlisting}[language=TeX]
% LaTeX Quine: Monster Walk 15 Layers
% Self-constructing document that folds 71→59→47→... symmetries

\documentclass{article}
\usepackage{amsmath}
\usepackage{listings}

\begin{document}

\title{Monster Walk: 15-Layer Quine}
\author{Generated from 196883 Shards}
\maketitle

\section*{Layer 0: Monster (196883 = 71×59×47)}

\begin{equation}
M = \bigotimes_{i=0}^{196883} V_i
\end{equation}

\section*{Layer 1: Split into 71 Shards}

\begin{equation}
\text{Shard}_{71}(i) = i \bmod 71, \quad i \in [0, 196883)
\end{equation}

\section*{Layer 2: Each Shard Splits into 59}

\begin{equation}
\text{Shard}_{59}(i) = \lfloor i / 71 \rfloor \bmod 59
\end{equation}

\section*{Layer 3: Each Splits into 47}

\begin{equation}
\text{Shard}_{47}(i) = \lfloor i / (71 \times 59) \rfloor \bmod 47
\end{equation}

\section*{Layer 4-15: Recursive Folding}

\begin{align}
L_4 &: 47 \to 43 \text{ (next prime)} \\
L_5 &: 43 \to 41 \\
L_6 &: 41 \to 37 \\
L_7 &: 37 \to 31 \\
L_8 &: 31 \to 29 \\
L_9 &: 29 \to 23 \text{ (Paxos nodes)} \\
L_{10} &: 23 \to 19 \\
L_{11} &: 19 \to 17 \\
L_{12} &: 17 \to 13 \\
L_{13} &: 13 \to 11 \\
L_{14} &: 11 \to 7 \\
L_{15} &: 7 \to 5 \to 3 \to 2 \to 1 \to 0
\end{align}

\section*{Quine Property}

This document source code:

\begin{lstlisting}[language=TeX]
% LaTeX Quine: Monster Walk 15 Layers
% Self-constructing document that folds 71→59→47→... symmetries

\documentclass{article}
\usepackage{amsmath}
\usepackage{listings}

\begin{document}

\title{Monster Walk: 15-Layer Quine}
\author{Generated from 196883 Shards}
\maketitle

\section*{Layer 0: Monster (196883 = 71×59×47)}

\begin{equation}
M = \bigotimes_{i=0}^{196883} V_i
\end{equation}

\section*{Layer 1: Split into 71 Shards}

\begin{equation}
\text{Shard}_{71}(i) = i \bmod 71, \quad i \in [0, 196883)
\end{equation}

\section*{Layer 2: Each Shard Splits into 59}

\begin{equation}
\text{Shard}_{59}(i) = \lfloor i / 71 \rfloor \bmod 59
\end{equation}

\section*{Layer 3: Each Splits into 47}

\begin{equation}
\text{Shard}_{47}(i) = \lfloor i / (71 \times 59) \rfloor \bmod 47
\end{equation}

\section*{Layer 4-15: Recursive Folding}

\begin{align}
L_4 &: 47 \to 43 \text{ (next prime)} \\
L_5 &: 43 \to 41 \\
L_6 &: 41 \to 37 \\
L_7 &: 37 \to 31 \\
L_8 &: 31 \to 29 \\
L_9 &: 29 \to 23 \text{ (Paxos nodes)} \\
L_{10} &: 23 \to 19 \\
L_{11} &: 19 \to 17 \\
L_{12} &: 17 \to 13 \\
L_{13} &: 13 \to 11 \\
L_{14} &: 11 \to 7 \\
L_{15} &: 7 \to 5 \to 3 \to 2 \to 1 \to 0
\end{align}

\section*{Quine Property}

This document source code:

\begin{lstlisting}[language=TeX]
\input{monster-walk-quine.tex}
\end{lstlisting}

\section*{Self-Reference}

\begin{equation}
Q(Q) = Q
\end{equation}

Where $Q$ is this document itself.

\end{document}

\end{lstlisting}

\section*{Self-Reference}

\begin{equation}
Q(Q) = Q
\end{equation}

Where $Q$ is this document itself.

\end{document}

\end{lstlisting}

\section*{Self-Reference}

\begin{equation}
Q(Q) = Q
\end{equation}

Where $Q$ is this document itself.

\end{document}

\end{lstlisting}

\section*{Self-Reference}

\begin{equation}
Q(Q) = Q
\end{equation}

Where $Q$ is this document itself.

\end{document}
